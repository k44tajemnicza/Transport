\subsection{Podsumowanie}
Równania Maxwella w postaci makroskopowej (w ośrodkach materialnych) mają postać:
\begin{equation} \nabla \cdot \vec{D}\arg=\wp\arg \label{r1}
 \end{equation}
  
  \begin{equation} \nabla \times \vec{H}\arg-\partial_t\vec{D}\arg=\vec{J}\arg \label{r2}
  \end{equation}

  \begin{equation} \nabla \times \E+\partial_t \B=\vec{0} \label{r3}\end{equation} 

  \begin{equation} \nabla \cdot \B=0 \end{equation}
  gdzie $\wp$ oznacza makroskopową gęstość ładunku, zdefiniowaną poprzednio jako: $\wp=<q_e\sum_i\delta(\r-\r_i(t))>+<\sum_nq_n\delta(\r-\r_n(t))$

  wn.1. Makroskopowe pola $\E,\B$ są wartościami średnimi pól mikroskopowych $\vec{e},\vec{b}$. Są to pola pierwotne, natomiast pola $\vec{D},\vec{H}$ są polami wtórnymi wynikającymi z ustalonej procedury średniowania.

  \subsection{Zasada zachowania ładunku}
  \subsubsection{Ogólne wyprowadzenie}
  Lokalnie (czyli w ośrodku) jest spełniona zasada zachowania ładunku, tzn. zmiana gęstości ładunku w ograniczonym obszarze $\Omega$ jest spowodowana przepływem prądu przez powierzchnię zamkniętą $\partial\Omega$ otaczającą ten obszar.

  (RYSUNEK WSTAW)

  \begin{equation}
  \frac{dQ}{dt}=-\int \vec{dS}\cdot\vec{J}\arg \label{zl}
  \end{equation}
  gdzie: 
  \begin{itemize}
  \item Q- całkowity ładunek, wyrażający się wzorem:
  \begin{equation}Q(t)=\int d^3r \rho\arg \label{Q} \end{equation}
  \item $\vec{dS}$  - wektor powierzchni, którego długość jest równa polu powierzchni, 
  \item natomiast wyrażenie po prawej stronie to natężenie prądu będące równe strumieniowi przepływającemu przez daną powierzchnię:
  \begin{equation}
  I(t)=\int \vec{dS}\cdot\vec{J}\arg 
  \end{equation}
  \end{itemize}
  uw. Minus w równaniu (\ref{zl}) oznacza, że ładunek może tylko wypływać spod powierzchni.\\
  uw2. Wyrażenie pod całką to strumień prądu płynący przez rozważany obszar.

  Wstawmy równanie (\ref{Q}) do równania (\ref{zl}):
  \begin{equation}
  \partial_t \int_{\partial\Omega} d^3r\rho\arg= -\int_{\partial\Omega}\vec{dS}\cdot\vec{J}\arg 
  \stackrel{\text{tw.Gaussa}}{=} -\int_\Omega d^3r\nabla\cdot\vec{J}\arg
  \end{equation}
  \begin{equation}
  \int d^3r\{\partial_t \rho\arg+\nabla \cdot\vec{J}\arg\}=0
  \end{equation}
  Stąd:
  \begin{equation}
  \partial_t \rho\arg+\nabla \cdot\vec{J}\arg=0 \label{zl2} \end{equation}
  Wzór (\ref{zl2}) to prawo zachowania ładunku - ładunek nie może zniknąć, może tylko przepłynąć przez powierzchnię.

  \subsubsection{Wyprowadzenie praw zachowania ładunku z praw Maxwella}
  Zadziałajmy obustronnie $\partial_t$ na 1. równanie Maxwella (\ref{r1}) oraz $\nabla\cdot$ na 2. równanie Maxwella (\ref{r2}):
  \begin{equation}
  (1)~~\Rightarrow ~~\partial_t \nabla\cdot \vec{D}\arg=\delta_t \rho(\r,t) ~~\Rightarrow~~ \nabla\cdot[\partial_t\vec{D}\arg=\partial_t\rho\arg \end{equation}
   \begin{equation}
    (2)~~\Rightarrow~~ \underbrace{\nabla\cdot[\nabla\times\vec{H}\arg]}_{=0 \text{ (bo jest to div z rot)}}-\nabla\cdot\partial_t\vec{D}\arg=\nabla\cdot\vec{J}\arg
	\end{equation}
	Łącząc oba te równania dostajemy:
	\begin{equation}
	-\partial_t\rho\arg=\nabla\cdot\vec{J}\arg \label{zz}
	\end{equation}
	Równanie (\ref{zz}) to zasada zachowania ładunku.

	\subsubsection{Równania materiałowe} 
	Z jednej strony równania Maxwella są niezmiennicze względem zmiany ośrodka, z drugiej strony ich rozwiązania- pola $\E,\B$- są różne w różnych ośrodkach. Dlatego potrzebujemy dodatkowych równań, które będą określać ośrodek- dlatego postulujemy równania materiałowe:
	\begin{equation} {D}_i\arg=\sum_{j/1}^3\int d^3r'\int_{-\infty}^t dt' \epsilon_{ij}\argg E_j \label{rm1}\end{equation}
	\begin{equation} H_i \arg =\sum_{j/1}^3\int d^3r'\int_{-\infty}^t dt' \mu^{-1}_{ij}\argg B_j \label{rm2}\end{equation}
	\begin{equation} J_i \arg =\sum_{j/1}^3\int d^3r'\int_{-\infty}^t dt' \sigma_{ij}\argg E_j \label{rm3}  \text{ - mikroskopowe prawo Ohma}\end{equation}
	\begin{verse}\textbf{wn.1.} Mamy zatem zestaw równań: Równania Maxwella+równania materiałowe \end{verse}
	\begin{verse}\textbf{wn.2.} W równaniach materiałowych jądrem całkowym są:\\
	(\ref{rm1}): ~ $\epsilon_{ij}\argg$ - to element tensora przenikalności elektrycznej ośrodka\\
	(\ref{rm2}):  ~$\mu^{-1}_{ij}\argg$ - to element tensora odwrotności przenikalności magnetycznej\\
	(\ref{rm3}):  ~$\sigma_{ij}\argg$  - to element tensora przewodnictwa elektrycznego. \end{verse}
	\begin{verse} \textbf{uw.1.} Równania materiałowe mają swoje uzasadnienie w termodynamice stanów nierównowagowych, natomiast do elektrodynamiki zostały dodane \textsl{ad hoc}.
	uw.2. \end{verse}
	\begin{verse} \textbf{uw.2.} Ostatnie (\ref{rm3}) równanie to mikroskopowe (lokalne) prawo Ohma, które można również zapisać w popularniejszej wersji:
	\begin{equation} \vec{J} \arg =\sigma\arg E\arg \end{equation} \end{verse}

	\subsubsection{Równania Maxwella a prąd stały}
	\begin{verse} \textbf{Zał.} Załóżmy, że \textbf{prąd jest stały}, tzn. płynie w sposób ciągły i nie gromadzi się (jest stały w czasie). \end{verse}

	Wówczas:\begin{itemize}
	\item Równanie Maxwella (\ref{r2}) $\Rightarrow$ powstaje stałe pole $\vec{H}$
	\item Równanie Maxwella (\ref{r3}) $\Rightarrow ~~ \nabla\times\vec{E}(\r)+\underbrace{\partial_t\vec{B}(\r)}_{=0}=0$ \\
	Stąd:
	\begin{equation} \nabla\times\vec{E}(\r)=0 \end{equation}
	Ponieważ wiemy, że dywergencja z rotacji daje 0, to $\vec{E}$ musi dać się przedstawić jako:
	\begin{equation} \vec{E}=-\nabla V(\r) \label{Epot}\end{equation}
	gdzie $V(\r)$ to potencjał.
	\begin{verse} \textbf{wn.} Jeśli prąd jest stały, to pole elektryczne ma potencjał. \end{verse} 
	\item Prawo zachowania ładunku (\ref{zl2}) $\Rightarrow ~~ \underbrace{\partial_t\rho\arg}_{=0}+\nabla\cdot\vec{J}\arg=0 $
	\begin{equation} \nabla\cdot\vec{J}({\r})=0 \end{equation}
	\item Mikroskopowe prawo Ohma $\Rightarrow ~~ \nabla[\sigma(\r)\vec{E}(\r)]=0$ \\
	Łącząc to równanie z równaniem (\ref{Epot}), dostajemy:
	\end{itemize}

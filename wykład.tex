\documentclass[11pt]{article}
%\renewcommand{\thesection}{\Roman{section}}  %zmiana section na rzymskie
\usepackage{amsmath, amsfonts, amsthm, amssymb}   % matematyka
\usepackage[utf8]{inputenc}
\usepackage{polski}
\usepackage[margin=60pt]{geometry}
%\usepackage[usenames,dvipsnames,table,xcdraw]{xcolor} % kolorowanie 
\usepackage{array} % taka tabelka dobra do macierzy
\usepackage{multirow}
\usepackage{hyperref} % linkowanie
\usepackage{relsize}
%%%%%%%%%%%%%%%%%%%%%%%%%%%%%%%%%%%%%%%%%%%%%%%%%%%%%%%%%%%%%%%%%%%%%%%%%%%%%%%%%%
% 
%%%%%%%%%%%%%%%%%%%%%%%%%%%%%%%%%%%%%%%%%%%%%%%%%%%%%%%%%%%%%%%%%%%%%%%%%%%%%%%%%%

%\newcommand{\ul}[1]{\ensuremath{\underline{#1}}}
\renewcommand{\r}{\vec{r}}
\renewcommand{\j}{\vec{j}}
\renewcommand{\v}{\vec{v}}
\newcommand{\B}{\vec{B}(\vec{r},t)}
\newcommand{\E}{\vec{E}(\vec{r},t)}
\renewcommand{\arg}{(\vec{r},t)}
\newcommand{\argg}{(\vec{r},\vec{r}',t,t')}
\newcommand{\ave}[1]{\left< #1 \right>}
\newcommand{\Brho}{\mathlarger{\mathlarger{\rho}}}
\title{Wstęp do kwantowej teorii transportu elektronowego}
\author{Sylwia Gołąb, Paweł Rzońca}

\begin{document}

\maketitle

\tableofcontents
\newpage


%%%%%%%%%%%%%%%%%%%%%%%%%%%%%%%%%%%%%%%%%%%%%%%%%%%%%%%%%%%%%%%%%%%%%%%%%%%%%%%%%%
% new subsection
%%%%%%%%%%%%%%%%%%%%%%%%%%%%%%%%%%%%%%%%%%%%%%%%%%%%%%%%%%%%%%%%%%%%%%%%%%%%%%%%%%
\section{Początki teorii elektronowej (subiektywnie)}
\begin{table}[h!]
    \centering
    \begin{tabular}{ll|ll|ll}
        \multicolumn{2}{c}{Elektrodynamika} & 
		\multicolumn{2}{|c|}{Teoria kinetyczna} & 
		\multicolumn{2}{c}{Teoria kwantowa} \\
            &&   1803 r. J. Dalton: & atomy   &&     \\
        1822 r. H. Davy: & $\sigma \sim S/L$ &&    &&     \\
        1826 r. G. Ohm: & $I \sim V$         & 1827 r. R. Brown: & ruchy  && \\
        1845 r. G. Kirchhoff: & $ j \sim E_f$& &&& \\
        1861 r. J. Maxwell: & równania & 1860 r. J. Maxwell: & rozkład $v$ && \\
        && 1865 r. J. Loschmidt: & rozmiar at. && \\
        && 1867 r. J. Maxwell: & równanie && \\
        && \multicolumn{2}{r|}{ciągłości o strukturze r. kinet.} && \\
        && 1872 r. L. Boltzmann: & równanie && \\
        1881 r. Helmholtz: & \multirow{2}{*}{elektron} &&&&\\
        Johnstone Stoney: &  &&&& \\
        1897 r. J. J. Thompson && 1900 r. D. Hilbert && 1900 r. M. Planck & \\
	&& 1905 r. Einstein i  & teoria r. && \\
	&& Smoluchowski: & Browna && \\ 
	1908 r. R. Millikan:& wart. $e$ &&&& \\ 
	1910 r. E. Rutherford:& budowa at. &&&&\\
	&& 1913 r. Bohr:& model at. && \\
	1916 r. Tolman-Steward:& bezwł. el. &&&&\\
	&&&& 1924 r. L. de Broglie & \\
	&&&& 1926 r. E. Schr\"{o}dinger & \\
	&&&& 1927 r. Fermi i Dirac: & stat. kw. \\
    \end{tabular}
\end{table}
Elektronowa teoria meterii 
\begin{itemize}
	\item[1845 r.] G. Fechner - Model prądu elektronowego 
	\item[1846 r.] W. Weber - Elektrodynamika cząstek
		$$ F = \dfrac{q_1q_2}{r^2} \left\{ 1 + \dfrac{r}{c^2} \ddot{r} (t) -
		\dfrac{1}{2c^2} \left[ \dot{r} (t) \right]^2 \right\} $$
	\item[1881 r.] Helmholtz
	\item[1897 r.] H. A. Lorentz - teoria elektronowa
	\item[1898 r.] E. Riecke - 
	\item[1900 r.] Drude - model przewodnictwa
	\item[1927 r.] Sommerfeld A. - statystyki kwantowe do opisu elektronów
	\item[1928 r.] Block
\end{itemize}
Teorie na przestrzeni czasu:
\begin{itemize}
	\item[1900 $\div$ 1927] Klasyczna teoria transportu elektronowego 
	\item[1927 $\div$ 1928] Półklasyczna teoria transportu elektronowego
	\item[1928 $\div$ 1933] Współczesna teoria transportu elektronowego 
\end{itemize}

%%%%%%%%%%%%%%%%%%%%%%%%%%%%%%%%%%%%%%%%%%%%%%%%%%%%%%%%%%%%%%%%%%%%%%%%%%%%%%%%%%
% new subsection
%%%%%%%%%%%%%%%%%%%%%%%%%%%%%%%%%%%%%%%%%%%%%%%%%%%%%%%%%%%%%%%%%%%%%%%%%%%%%%%%%%
\section{Teoria elektronowa Lorenza}
\textbf{Założenia:}
\begin{enumerate}
	\item Ośrodki materiale mają strukturę dyskretną, tzn. zbudowane są z 
		cząstek naładowanych, które w sumie dają układ neutralny.
	\item Wszystkie zjawiska w ośrodku materialnym są spowodowane ruchem 
		cząstek naładowanych pod wpływem pól zewnętrznych, przy czym:
		\begin{enumerate}
			\item w dielektrykach cząstki naładowane są związane i mogą
				wykonywać drgania wokół położeń równowagi lub ulegać
				nieznacznym wychyeniom pod wpływem przyłożonego $\vec{E}$,
			\item w przewodnikach prócz cząstek związanych występują także
				czastki naładowane swobodne, których ruch powoduje 
				powstanie prądu elektrycznego,
			\item w ośrodkach magnetycznych istnieją cząstki naładowane 
				posiadające wewnętrzny moment magnetyczny lub niezerowy
				moment pędu.
		\end{enumerate}
	\item Mikroskopowe pola elektromagnetyczne wytwarzane przez cząstki 
		naładowane tworzące rozpatrywany ośrodek są rozwiązaniami 
		równań Maxwella w próżni:\\
	\begin{equation}{\label{Max_mikro}}
	\left\{ 
		\begin{array}{l}
		\nabla \circ \vec{e} (\vec{r},t) = \rho(\vec{r},t)\\
		\nabla \times \vec{b}(\vec{r},t) - \partial_t \vec{e}(\vec{r},t) 
			= \vec{j}(\vec{r},t)\\
		\nabla \times \vec{e} (\vec{r},t) + \partial_t \vec{b} (\vec{r},t)
			= \vec{0} \\
		\nabla \circ \vec{b} (\vec{r},t) = 0.
		\end{array}
	\right.
	\end{equation}
	$\vec{e}(\vec{r},t), \ \vec{b}(\vec{r},t)$  - mikroskopowe pola elektryczne i
		magnetyczne \\
	$\rho (\vec{r},t)= \sum_i q_i \delta (\vec{r} -  \vec{r_i} (t))$\\
	$\vec{j}(\vec{r},t)= \sum_i \vec{v_i} (t) \delta(\vec{r} - \vec{r_i}(t))$
	\item Gęstość siły działająca na $\vec{\rho}(\vec{r},t)$ ma postać
	$$ \vec{f}(\vec{r},t) = \vec{\rho}(\vec{r},t) [ \vec{e}(\vec{r},t)+ \vec{v}(t)
		\times \vec{b}(\vec{r},t)] $$
	$$ \vec{F} (t) = \int d^3r' f(\vec{r}\ ',t) = $$
	przy założeniu jednorodności $\vec{b}$ i $\vec{e}$
	$$ = \int d^3r' \{ \rho \arg [ \vec{e} + \vec{v}(t) \times \vec{b} ]  \} =
	\int d^3r' \{ q \delta (\vec{r} - \vec{r}\ ') [ \vec{e} + \vec{v}(t) 
	\times \vec{b} ] \} = q[ \vec{e} + \vec{v}(t) \times \vec{b} ] \int d^3r' 
	\delta (\vec{r} - \vec{r} \ ').$$
	Ostatecznie
	\begin{equation}
		\vec{F} = q (\vec{e} + \vec{v} \times \vec{b})
	\end{equation}
	\begin{equation}
		m\ddot{\vec{r}} (t) = q[ \vec{e} + \vec{v} (t) \times \vec{b} ]. 
	\end{equation}
\end{enumerate}
Zmiany przestrzenne $\vec{e} \arg$ i $\vec{b} \arg $ są znaczące na odcinkach
rzędu $10^{-10} \mbox{m} = 1 \stackrel{\circ}{\mbox{A}} = 0,1 \mbox{nm}.$\\
Zmiany czasowe są rzędu $10^{-13} \div 10^{-17}$s. 
\\
Klasyczny promień elektronu $r_e = \frac{1}{4\pi \epsilon_0} \frac{e^2}{mc^2} 
\approx 2,82 \cdot 10^{-6}$nm, rozmiar protonu $r_p \approx 0,88 \cdot 
10^{-6}$nm natomiast promień atomu $r_p \approx 0,1$nm.

%%%%%%%%%%%%%%%%%%%%%%%%%%%%%%%%%%%%%%%%%%%%%%%%%%%%%%%%%%%%%%%%%%%%%%%%%%%%%%%%%%
% new section
%%%%%%%%%%%%%%%%%%%%%%%%%%%%%%%%%%%%%%%%%%%%%%%%%%%%%%%%%%%%%%%%%%%%%%%%%%%%%%%%%%
\section{Makroskopowa elektrodynamika ośrodków materialnych}
\textbf{Hipotezia: }
Makroskopowe pola $\vec{E}$ i $\vec{B}$ są wartościami średnimi pól 
mikroskopowych$\vec{e}$ i $\vec{b}$.
\begin{equation}
	\vec{E} \arg = \left< \vec{e} \arg \right>
\end{equation}
\begin{equation}
	\vec{B} \arg = \left< \vec{b} \arg \right>,
\end{equation}
gdzie średnia jest przestrzenna, czyli
$$ \left< \vec{f} \arg \right> \equiv \int d^3 r' w(\vec{r}\ ')\vec{f}
( \vec{r} - \vec{r}\ ',t). $$
$w(\vec{r}\ ')$ - funkcja wagowa spełniająca warunki:
\begin{enumerate}
	\item jest funkcją rzeczywistą dodatnio określoną,
	\item jest znormalizowana $$\int_{\Omega} d^3 r' w(\vec{r}\ ') = 1,$$
	\item jest wolnozmienna, tj.
		$$w(\vec{r}\ '+\vec{a}) = \sum_n \frac{1}{n!} \left[ \vec{a} \nabla 
		\right]^n w(\vec{r})_{\big|_{\vec{r}=\vec{r}'}}$$
		$$w(\vec{r}\ '+\vec{a}) = w(\vec{r}\ ')\pm[\vec{a}\nabla] 
		w(\vec{r}\ ') +\frac{1}{2} [\vec{a}\nabla]^2w(\vec{r}\ '),$$
	\item rozciągłość duża w porównaniu z wielkością cząstek.
\end{enumerate}

RYSUNEK
\subsection{Wyprowadzenie makroskopowych praw Maxwella z 
mikroskopowych odpowiedników}
Zgodnie z równaniami mikroskopowymi \ref{Max_mikro}:
\begin{equation} 
	\nabla \cdot \vec{E}\arg=\left< \rho \arg \right> \label{startr1}
\end{equation}
\begin{equation} 
	\nabla \times \vec{B}\arg-\partial_t\vec{E}\arg=\left< \vec{j}\arg
	\right> \label{startr2}
\end{equation}
\begin{equation} 
	\nabla \times \E+\partial_t \B=\vec{0} \label{startr3}
\end{equation}
\begin{equation} 
	\nabla \cdot \B=0 \label{startr4}
\end{equation}
\begin{center} RYSUNEK \end{center}
Najpierw obliczymy średnią z gęstości ładunków.
Gęstość ładunku można rozbić na gęstość ładunków swobodnych oraz 
gęstość ładunków związanych
$$\rho \arg = \rho_{free} \arg + \rho_{bound} \arg$$
gdzie:\\
$\rho_{free} \arg = q_e \sum\limits_i \delta (\vec{r}-\vec{r}_i(t))$\\
$\rho_{bound} \arg = \sum\limits_n \underbrace{\rho_n\arg}_{n-tego\
jonu}  = \sum\limits_n \sum\limits_j q_{jn} \delta (\vec{r}-\vec{r}_j(t)) 
= \sum\limits_{n} \sum\limits_{j} g_{jn}\delta(\vec{r}-\vec{r}_n(t)-
\vec{r}_{jn}(t)).$\\
$$\ave{ \rho \arg } = \ave{\rho_{free}\arg} + \ave{\rho_{bound}\arg} = $$
$=\int d^3r' w(\vec{r}\ ') \rho_{free}(\vec{r}-\vec{r}_j\ '(t))+
\int d^3r' w(\vec{r}\ ') \rho_{bound}(\vec{r}-\vec{r}_j\ '(t))=$\\
$ = \int d^3r' w(\vec{r}\ ')q_e \sum\limits_{i}\delta (\vec{r}-\vec{r}_i(t)
-\vec{r}\ ') + \int d^3r' w(\vec{r}\ ') \sum\limits_n \sum\limits_j q_{jn}
\delta (\vec{r}-\vec{r}_j\ '(t) - \vec{r}\ ')= $\\
$ = q_e \sum\limits_i w(\vec{r} - \vec{r}_i(t)+ \sum\limits_n \sum\limits_j q_{in} 
w(\vec{r}-\vec{r}_n(t)-\vec{r}_{jn}(t)=(*) .$
Z własności $w$ wiemy, że:
$$w(\vec{r}-\vec{r}_n(t)-\vec{r}_{jn}(t))\simeq w(\r-\r_n(t))-
[\r_{jn}\cdot\nabla]w(\r-\r_n(t)).$$
$(*)=  q_e \sum\limits_i w(\vec{r} - \vec{r}_i(t))+ \sum\limits_n 
\sum\limits_j q_{in} 
[w(\r-\r_n(t))-[\r_{jn}\cdot\nabla]w(\r-\r_n(t))]$\\
Całkowity ładunek jonu: $q_n=\sum\limits_j q_{jn}$.\\
Moment dipolowy $\vec{d}_n(t)=\sum\limits_j d_{jn}(t) = 
\sum\limits_j q_{jn}\vec{r}_{jn}(t).$\\
$$\ave{ \rho \arg } = q_e\sum_i w(\r-\r_i(t))+\sum_n q_n w(\r-\r_n(t))-\nabla\cdot
\sum_nw(\r-\r_n(t))\vec{d}_n$$
$$\ave{ \rho \arg } =\underbrace{\ave{ q_e\sum_i \delta(\r-\r_i(t))}+\ave{\sum_n q_n 
\delta(\r-\r_n(t))}}_{\mbox{makroskopowa gęstość ładunku}}-\nabla\cdot
\underbrace{\ave{\sum_n\delta(\r-\r_n(t))\vec{d}_n(t)}}_{\mbox{makroskopowa
polaryzacja}}$$
$$\ave{\rho\arg}=\Brho\arg - \nabla\cdot\vec{P}\arg.$$
Wracając do równania \ref{startr1}
$$\nabla\cdot\vec{E}\arg=\ave{\rho\arg}=\Brho\arg-\nabla\vec{P}\arg$$
$$\nabla\cdot(\vec{E}\arg+\nabla\vec{P}\arg)=\Brho\arg$$
$$ \vec{E}\arg+\nabla\vec{P}\arg\equiv\vec{D}\arg$$
gdzie $\vec{D}\arg$ - wektor indukcji elektrycznej
$$D_i\arg=\sum_{k/1}^{3}\int d^3r\int_{-\infty}^{t}dt'\epsilon_{kj}(\vec{r},
\vec{r}\ ',t,t')E_j(\vec{r}\ ',t')$$
$$D_i=\sum_{k/1}^3\epsilon_{kj}E_j.$$

\section{Makroskopowa elektrodynamika ośrodków materialnych}
\subsection{Podsumowanie}
Równania Maxwella w postaci makroskopowej (w ośrodkach materialnych) mają postać:
\begin{equation} \nabla \cdot \vec{D}\arg=\wp\arg \label{r1}
 \end{equation}
 
\begin{equation} \nabla \times \vec{H}\arg-\partial_t\vec{D}\arg=\vec{J}\arg \label{r2}
\end{equation}

\begin{equation} \nabla \times \E+\partial_t \B=\vec{0} \label{r3}\end{equation} 

\begin{equation} \nabla \cdot \B=0 \end{equation}
gdzie $\wp$ oznacza makroskopową gęstość ładunku, zdefiniowaną poprzednio jako: $\wp=<q_e\sum_i\delta(\r-\r_i(t))>+<\sum_nq_n\delta(\r-\r_n(t))$

wn.1. Makroskopowe pola $\E,\B$ są wartościami średnimi pól mikroskopowych $\vec{e},\vec{b}$. Są to pola pierwotne, natomiast pola $\vec{D},\vec{H}$ są polami wtórnymi wynikającymi z ustalonej procedury średniowania.

\subsection{Zasada zachowania ładunku}
\subsubsection{Ogólne wyprowadzenie}
Lokalnie (czyli w ośrodku) jest spełniona zasada zachowania ładunku, tzn. zmiana gęstości ładunku w ograniczonym obszarze $\Omega$ jest spowodowana przepływem prądu przez powierzchnię zamkniętą $\partial\Omega$ otaczającą ten obszar.
\begin{center}
\includegraphics[width=6cm] {obrazek1}
\end{center}
\textit{Rys. 1. Rysunek pomocniczy.}
Spełnione jest:
\begin{equation}
\frac{dQ}{dt}=-\int \vec{dS}\cdot\vec{J}\arg \label{zl}
\end{equation}
gdzie: 
\begin{itemize}
\item $\vec{dS}$ - element powierzchni; $|\vec{dS}|$ - pole powierzchni
\item Q- całkowity ładunek, wyrażający się wzorem:
\begin{equation}Q(t)=\int d^3r \rho\arg \label{Q} \end{equation}
\item $\vec{dS}$  - wektor powierzchni, którego długość jest równa polu powierzchni, 
\item natomiast wyrażenie po prawej stronie to natężenie prądu będące równe strumieniowi przepływającemu przez daną powierzchnię:
\begin{equation}
I(t)=\int \vec{dS}\cdot\vec{J}\arg 
\end{equation}
\end{itemize}
uw. Minus w równaniu (\ref{zl}) oznacza, że ładunek może tylko wypływać spod powierzchni.\\
uw2. Wyrażenie pod całką to strumień prądu płynący przez rozważany obszar.

Wstawmy równanie (\ref{Q}) do równania (\ref{zl}):
\begin{equation}
\partial_t \int_{\partial\Omega} d^3r\rho\arg= -\int_{\partial\Omega}\vec{dS}\cdot\vec{J}\arg 
\stackrel{\text{tw.Gaussa}}{=} -\int_\Omega d^3r\nabla\cdot\vec{J}\arg
\end{equation}
\begin{equation}
\int d^3r\{\partial_t \rho\arg+\nabla \cdot\vec{J}\arg\}=0
\end{equation}
Stąd:
\begin{equation}
\partial_t \rho\arg+\nabla \cdot\vec{J}\arg=0 \label{zl2} \end{equation}
Wzór (\ref{zl2}) to prawo zachowania ładunku - ładunek nie może zniknąć, może tylko przepłynąć przez powierzchnię.

\subsubsection{Wyprowadzenie praw zachowania ładunku z praw Maxwella}
Zadziałajmy obustronnie $\partial_t$ na 1. równanie Maxwella (\ref{r1}) oraz $\nabla\cdot$ na 2. równanie Maxwella (\ref{r2}):
\begin{equation}
(1)~~\Rightarrow ~~\partial_t \nabla\cdot \vec{D}\arg=\delta_t \rho(\r,t) ~~\Rightarrow~~ \nabla\cdot[\partial_t\vec{D}\arg=\partial_t\rho\arg \end{equation}
 \begin{equation}
 (2)~~\Rightarrow~~ \underbrace{\nabla\cdot[\nabla\times\vec{H}\arg]}_{=0 \text{ (bo jest to div z rot)}}-\nabla\cdot\partial_t\vec{D}\arg=\nabla\cdot\vec{J}\arg
\end{equation}
Łącząc oba te równania dostajemy:
\begin{equation}
-\partial_t\rho\arg=\nabla\cdot\vec{J}\arg \label{zz}
\end{equation}
Równanie (\ref{zz}) to zasada zachowania ładunku.

\subsubsection{Równania materiałowe} 
Z jednej strony równania Maxwella są niezmiennicze względem zmiany ośrodka, z drugiej strony ich rozwiązania- pola $\E,\B$- są różne w różnych ośrodkach. Dlatego potrzebujemy dodatkowych równań, które będą określać ośrodek- dlatego postulujemy równania materiałowe:
\begin{equation} {D}_i\arg=\sum_{j/1}^3\int d^3r'\int_{-\infty}^t dt' \epsilon_{ij}\argg E_j \label{rm1}\end{equation}
\begin{equation} H_i \arg =\sum_{j/1}^3\int d^3r'\int_{-\infty}^t dt' \mu^{-1}_{ij}\argg B_j \label{rm2}\end{equation}
\begin{equation} J_i \arg =\sum_{j/1}^3\int d^3r'\int_{-\infty}^t dt' \sigma_{ij}\argg E_j \label{rm3}  \text{ - mikroskopowe prawo Ohma}\end{equation}
\begin{verse}\textbf{wn.1.} Mamy zatem zestaw równań: Równania Maxwella+równania materiałowe \end{verse}
\begin{verse}\textbf{wn.2.} W równaniach materiałowych jądrem całkowym są:\\
(\ref{rm1}): ~ $\epsilon_{ij}\argg$ - to element tensora przenikalności elektrycznej ośrodka\\
(\ref{rm2}):  ~$\mu^{-1}_{ij}\argg$ - to element tensora odwrotności przenikalności magnetycznej\\
(\ref{rm3}):  ~$\sigma_{ij}\argg$  - to element tensora przewodnictwa elektrycznego. \end{verse}
\begin{verse} \textbf{uw.1.} Równania materiałowe mają swoje uzasadnienie w termodynamice stanów nierównowagowych, natomiast do elektrodynamiki zostały dodane \textsl{ad hoc}.
uw.2. \end{verse}
\begin{verse} \textbf{uw.2.} Ostatnie (\ref{rm3}) równanie to mikroskopowe (lokalne) prawo Ohma, które można również zapisać w popularniejszej wersji:
\begin{equation} \vec{J} \arg =\sigma\arg E\arg \end{equation} \end{verse}

\subsubsection{Równania Maxwella a prąd stały}
\begin{verse} \textbf{Zał.} Załóżmy, że \textbf{prąd jest stały}, tzn. płynie w sposób ciągły i nie gromadzi się (jest stały w czasie). \end{verse}

Wówczas:\begin{itemize}
\item Równanie Maxwella (\ref{r2}) $\Rightarrow$ powstaje stałe pole $\vec{H}$
\item Równanie Maxwella (\ref{r3}) $\Rightarrow ~~ \nabla\times\vec{E}(\r)+\underbrace{\partial_t\vec{B}(\r)}_{=0}=0$ \\
Stąd:
\begin{equation} \nabla\times\vec{E}(\r)=0 \end{equation}
Ponieważ wiemy, że dywergencja z rotacji daje 0, to $\vec{E}$ musi dać się przedstawić jako:
\begin{equation} \vec{E}=-\nabla V(\r) \label{Epot}\end{equation}
gdzie $V(\r)$ to potencjał.
\begin{verse} \textbf{wn.} Jeśli prąd jest stały, to pole elektryczne ma potencjał. \end{verse} 
\item Prawo zachowania ładunku (\ref{zl2}) $\Rightarrow ~~ \underbrace{\partial_t\rho\arg}_{=0}+\nabla\cdot\vec{J}\arg=0 $
\begin{equation} \nabla\cdot\vec{J}({\r})=0 \end{equation}
\item Mikroskopowe prawo Ohma $\Rightarrow ~~ \nabla[\sigma(\r)\vec{E}(\r)]=0$ \\
Łącząc to równanie z równaniem (\ref{Epot}), dostajemy: \\ 
\begin{equation} -\nabla \cdot[\sigma(\r) \nabla V(\r)] =0  \nonumber \end{equation}
\begin{equation} \nabla \cdot[\sigma(\r) \nabla V(\r)] =0 \label{doLaplace} \end{equation}
\item Załóżmy teraz, że przewodnictwo jest wszędzie takie samo: $\sigma(\r)=const=\sigma $.\\
Wówczas z równania (\\ref{doLaplace}) wynika:
\begin{equation} \sigma \nabla ^2 V(\r) =0 \end{equation}
O ile $\sigma \neq 0 $ (czyli nie jest to izolator):
\begin{equation} \nabla^2 V(\r)=0\end{equation}
Jest to równania Laplace'a.
\begin{verse} \textbf{wn.} Jeśli prąd jest stały, to potencjał układu spełnia równanie Laplace'a. \end{verse}
\begin{verse} \textbf{uw.} Bez założenia o prądzie stałym dostalibyśmy równanie Poissona \end{verse}
\end{itemize}
\subsubsection{Dygresja - potencjał a energia potencjalna}
Energia potencjalna wyraża się wzorem:
\begin{equation} U(\r) \equiv \int d^3r'\rho(\r')V(\r) \end{equation}
Łącząc powyższe równanie z definicją gęstości ładunkowej:
\begin{equation} U(\r) = \int d^3r'q\delta(\r-\r')V(\r)\end{equation}
Stąd:
\begin{equation} U(\r) = qV(\r) \label{Pot}\end{equation}
Równanie (\ref{Pot}) to związek pomiędzy energią potencjalną a potencjałem.

\section{Zlinearyzowane relacje konstytutywne ośrodków materialnych}
\subsection{Ogólna postać równań materiałowych}
Można zauważyć, że wszystkie równania materiałowe (\ref{rm1}),(\ref{rm2}),(\ref{rm3}) mają postać:
\begin{equation} \vec{Y}(\r,t)=\int d^3r \int_{-\infty}^t dt' \hat{\chi}(\r,\r',t,t') \vec{X}(\r',t') \end{equation}
lub równoważnie:
\begin{equation} {Y}_i(\r,t)=\sum_{j/1}^3 \int d^3r \int_{-\infty}^t dt' {\chi}_{ij}(\r,\r',t,t') X_j(\r',t') \end{equation}
gdzie:\\
\begin{itemize} \item $\vec{Y}$ - wektor reprezentujący pole wtórne
\item $\vec{X}$ - wektor reprezentujący pole pierwotne
\item $\hat{\chi} $ - to tzw. uogólniona podatność (inaczej: funkcja odpowiedzi układu). Jest to tensorowe jądro całkowe, służące do przekształcenia pola pierwotnego we wtórne - zatem wnosi ona informację o ośrodku.
\end{itemize}
\begin{verse}\textbf{uw.}
Dlaczego całka po czasie biegnie do $t$ a nie do $\infty$? \\
Ponieważ wówczas $\chi$ zbiera informacje do chwili obecnej. Gdyby całka była do $\infty$, to złamalibyśmy \textbf{zasadę przyczynowości} (wyraża ona, że skutek obserwowany w chwili obecnej zależy tylko do przyczyn z przeszłości). \\
Można zatem postawić:
\begin{equation}\chi_{ij}(\r,\r',t,t')=0 \text{ ~~~~dla~ } t'>t\end{equation}
\end{verse}

\subsection{Równania materiałowe a teoria liniowej odpowiedzi}
\textbf{Fakty:}
\begin{enumerate}
\item Pola wtórne są liniowymi funkcjonałami pól pierwotnych:
\begin{equation} \vec{Y}[\alpha_1\vec{X}_1 + \alpha_2\vec{X}_2]=\alpha_1 \vec{Y}[\vec{X}_1] + \alpha_2 \vec{Y}[\vec{X}_2] \end{equation}
\textbf{uw.} Jeśli uciąglimy tę sumę, dostaniemy całkę.
\item Równania materiałowe pozostają słuszne, jeśli pola pierwotne można traktować jako słabe zaburzenia. Wówczas można rozwinąć w szereg McLaurina:
\begin{equation}\vec{Y}[\vec{X}]=\vec{Y}[\vec{0}]+ \frac{\delta\vec{Y}}{\delta\vec{X}}|_{\vec{0}}\vec{X} + \frac{1}{2}\frac{\delta^2 \vec{Y}}{\delta\vec{X}^2}|_{\vec{0}}\vec{X}^2 + ...\end{equation}
gdzie: $\vec{Y}[\vec{0}]=0$ (bo nie może istnieć pole wtórne bez pierwotnego), zatem:
\begin{equation}[\vec{X}]=\vec{Y}[\vec{0}]+ \frac{\delta\vec{Y}}{\delta\vec{X}}|_{\vec{0}}\vec{X} + \mathcal{O}(\vec{X}^2)\end{equation}
\end{enumerate}
\textbf{Założenie:}\begin{verse} 
Załóżmy, że pola wtórne są proporcjonalne do pól pierwotnych (tzw. linearyzacja równania). Wówczas:
\begin{equation}\vec{Y}[{\vec{X}}] \simeq \frac{\delta\vec{Y}}{\delta\vec{X}}|_{\vec{0}} \vec{X} = \hat{\mathcal{L}}\vec{X}\end{equation} \end{verse}
Ostatecznie więc:
\begin{equation}\vec{Y}[{\vec{X}}] = \hat{\mathcal{L}}\vec{X} \label{RFen} \end{equation}
To przybliżenie nazywamy \textbf{teorią liniowej odpowiedzi}, zaś samo równanie (\ref{RFen}) - równaniem fenomenologicznym, a współczynniki $\hat{\mathcal{L}}$ - współczynnikami fenomenologicznymi. Współczynniki te dostajemy z doświadczeń i następnie staramy się je wyjaśnić za pomocą teorii.

\subsection{Uogólnienie na wiele pól zaburzających - zjawiska krzyżowe}
Z racji liniowości wektora $\vec{Y}$, równanie (\ref{RFen}) można uogólnić na wiele pól zaburzających (np. możemy jednocześnie rozważać pola $\B$ i $\E$) :
\begin{equation} \vec{Y}=\hat{\mathcal{L}}\vec{X} \arg 
\stackrel{\text{uogólnienie}}{\longrightarrow}  {Y}_i=\sum_{j/1}^n \mathcal{L}_{ij} X_j \end{equation}
\begin{verse} \textbf{Np.} Niech n=2. Wówczas:
\begin{center} 
$\begin{cases} \vec{Y}_1=\mathcal{L}_{11}\vec{X}_1+\mathcal{L}_{12}\vec{X}_2 ~~~~~~~~~~/\mathcal{L}_{12}^{-1}\\ \vec{Y}_2=\mathcal{L}_{21}\vec{X}_1+\mathcal{L}_{22}\vec{X}_2 ~~~~~~~~~~/\mathcal{L}_{22}^{-1}
\end{cases}$
\end{center}

\textbf{wn.} Pole wtórne wynika z obu pól pierwotnych.\\
Wyznaczamy $\vec{X}_1$:
\begin{center}
$\begin{cases} \mathcal{L}_{12}^{-1}\vec{Y}_1=\mathcal{L}_{12}^{-1}\mathcal{L}_{11}\vec{X}_1+\vec{X}_2 \\ \mathcal{L}_{22}^{-1}\vec{Y}_2=\mathcal{L}_{22}^{-1}\mathcal{L}_{21}\vec{X}_1+\vec{X}_2 
\end{cases}$
\end{center}
Odejmując stronami, dostajemy:
\begin{equation} \mathcal{L}_{12}^{-1}\vec{Y}_1 - {L}_{22}^{-1}\vec{Y}_2 = [\mathcal{L}_{12}^{-1}\mathcal{L}_{11} - \mathcal{L}_{22}^{-1}\mathcal{L}_{21}]\vec{X}_1 \nonumber \end{equation}
\begin{equation} 
 [\mathcal{L}_{12}^{-1}\mathcal{L}_{11}-\mathcal{L}_{22}^{-1}\mathcal{L}_{21}]^{-1} \mathcal{L}_{12}^{-1}\vec{Y}_1 -   
 [\mathcal{L}_{12}^{-1}\mathcal{L}_{11} - \mathcal{L}_{22}^{-1}\mathcal{L}_{21}]^{-1} \mathcal{L}_{22}^{-1}\vec{Y}_2 =\vec{X}_1  \nonumber \end{equation}
 
\textbf{wn.} Pole pierwotne $\vec{X}_1$ można przedstawić w postaci kombinacji liniowej pól wtórnych, przy czym $\vec{X}_1$ produkuje $\vec{Y}_1$ oraz $-\vec{Y}_2$. Zauważmy, że $\vec{Y}_2$ jest z minusem, bo przeciwdziała ono polu $\vec{Y}_1$.\\ ~~~~Takie procesy z polami $\vec{Y}_1$ i $\vec{Y}_2$ naz. zjawiskami krzyżowymi.\\
\textbf{np.} W zjawiskach termoelektrycznych polami tymi są $\vec{E}$ i gradient temperatury $\nabla T$: Pole elektryczne przemieszcza elektrony, ale przez opór materiał się grzeje, więc powstaje gradient temperatury.
\end{verse}
\subsection{Klasyfikacja materiałów ze względu na jądro całkowe równania materiałowego}
\begin{enumerate}
\item Ośrodek materialny jest \textbf{lokalnie liniowy} wtedy i tylko wtedy, gdy $\hat{\chi}$ ma postać:
\begin{equation}\hat{\chi}(\r,\r',t,t')=\hat{\chi}(\r',t,t')\delta(\r-\r')\end{equation}
\item Ośrodek materialny jest \textbf{przestrzennie jednorodny} wtedy i tylko wtedy, gdy $\hat{\chi}$ ma postać:
\begin{equation}\hat{X}(\r,\r',t,t')=\hat{\chi}(\r-\r',t,t')\end{equation}
Jeśli równość ta nie zachodzi, to ośrodek jest \textbf{niejednorodny przestrzennie}.
\item Ośrodek materialny jest \textbf{czasowo jednorodny} wtedy i tylko wtedy, gdy $\hat{\chi}$ ma postać:
\begin{equation}\hat{\chi}(\r,\r',t,t')=\hat{\chi}(\r,\r',t-t')\end{equation}
 Np. gdy materiał się grzeje, to w różnych chwilach różne jest pole $\nabla T$
 Jeśli równość ta nie zachodzi, to ośrodek jest \textbf{niejednorodny czasowo}.
 \item Ośrodek materialny jest \textbf{czasowo i przestrzennie jednorodny} wtedy i tylko wtedy, gdy $\hat{\chi}$ ma postać:
\begin{equation}\hat{\chi}(\r,\r',t,t')=\hat{\chi}(\r-\r',t-t')\end{equation}
 Tę własność spełnia gaz elektronowy oraz nukleony w jądrze.
  \item Ośrodek materialny jest \textbf{izotropowy} wtedy i tylko wtedy, gdy elementy macierzowe $\hat{\chi}$ mają postać:
\begin{equation}{\chi}_{ij}(\r,\r',t,t')=\hat{\chi}(\r-\r',t-t')\delta_{ij}\end{equation}
 Jeśli równość ta nie zachodzi, to ośrodek jest \textbf{anizotropowy}.
 \item Ośrodek materialny jest \textbf{homogeniczny} wtedy i tylko wtedy, gdy elementy macierzowe $\hat{\chi}$ mają postać:
\begin{equation}{\chi}_{ij}(\r,\r',t,t')={\chi}\delta(\r-\r')\delta(t-t')\delta_{ij}\end{equation}
\end{enumerate}
\subsection{Równanie materiałowe dla ośrodka homogenicznego- konsekwencje}
Wróćmy do równania materiałowego:
\begin{equation} {Y}_i(\r,t)=\sum_{j/1}^3 \int d^3r \int_{-\infty}^t dt' {\chi}_{ij}(\r,\r',t,t') X_j(\r',t') \end{equation}
W ośrodku homogenicznym:
\begin{equation} {Y}_i(\r,t)=\sum_{j/1}^3 \int d^3r \int_{-\infty}^0 dt' {\chi}\delta(\r-\r')\delta(t-t')\delta_{ij} X_j(\r',t')\arg 
\stackrel{\text{tw.filtracyjne}}{=}\chi X_i(\r,t) \end{equation}
Zatem:\\
\begin{center}
%\caption{fsdF}
\begin{tabular}{|c||c|c|}
  \hline
 ${_Y^~~X}$ & $\vec{E}$ & $\vec{B}$\\
  \hline\hline
  $\vec{D}$ &  $\hat{\epsilon}$ & brak \\
\hline
  $\vec{H}$ &  brak & $\hat{\mu}^{-1}$ \\
  \hline
  $\vec{J}$ &  $\hat{\sigma}$ &  brak\\
    \hline
\end{tabular} 
\end{center}
\textbf{wn. 1.} Z powyższego wynika mikroskopowe prawo Ohma dla ośrodków homogenicznych: \begin{equation} \vec{J}(\r,t)=\sigma\E \end{equation}
Zatem Ohm miał szczęście, że przykładał małe pola (bo w powyższych rachunkach zastosowaliśmy rachunek zaburzeń prawdziwy dla małych pól).\\
\textbf{wn. 2.} Dla układów homogenicznych skalarna stała $\chi$ reprezentuje stałą materiałową, która opisuje w sposób ilościowy rozpatrywaną własność ośrodka.
\subsection{Punkt widzenia}
Ustalmy jeden z dwóch możliwych punktów widzenia:
prąd elektryczny jest konsekwencją przyłożonego pola elektrycznego $\E$. Pole elektryczne to przyczyna, a prąd to skutek.


\end{document}




\end{document}


